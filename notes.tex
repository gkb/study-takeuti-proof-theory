\documentclass{article}

\usepackage[utf8]{inputenc}
\usepackage{amsmath}
\usepackage{mathabx}
\usepackage{mathrsfs}
\usepackage{hyperref}
\usepackage{MnSymbol}
\usepackage[dvipsnames]{xcolor}
\usepackage{bussproofs}
\newtheorem{theorem}{Theorem}
\newtheorem{lemma}{Lemma}

\title{Notes from Proof Theory by Takeuti}
\author{gajukbhat}
\date{April 2022}

\begin{document}

\maketitle

\section{Chapter 1}
\subsection{Lemma 2.12}
Make the proof regular to get \(P''(a)\). If the proof contains no eigenvariables that are \(a\) or contained in \(t\), then \(P'(a) = P''(a)\). 
Let the induction hypothesis be that if the proof contains up to \(n\) eigenvariables that are \(a\) or contained in \(t\), then there's a \(P'(a)\) derived from \(P(a)\) that is a proof of \(\Gamma(a) \rightarrow \Delta(a)\). It's possible to show that the induction hypothesis implies that if the proof contains \(n + 1\) eigenvariables that are \(a\) or contained in \(t\), then there's a \(P'(a)\) derived from \(P(a)\) that is a proof of \(\Gamma(a) \rightarrow \Delta(a)\).
From that the conclusion follows.

\subsection{Definition 2.15}
I want to prove that the definition defines an equivalence relation. Suppose for some \(u_1, \ldots, u_n\) and \(v_1, \ldots, v_n\)
\[
A\biggl(\frac{u_1, \ldots, u_n}{w_1, \ldots, w_n}\biggr)
\]
and
\[
B\biggl(\frac{v_1, \ldots, v_n}{w_1, \ldots, w_n}\biggr)
\]
are the same.
\end{document}
