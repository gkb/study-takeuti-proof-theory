\documentclass{article}

\usepackage[utf8]{inputenc}
\usepackage{amsmath}
\usepackage{mathabx}
\usepackage{mathrsfs}
\usepackage{hyperref}
\usepackage{MnSymbol}
\usepackage[dvipsnames]{xcolor}
\usepackage{bussproofs}
\newtheorem{theorem}{Theorem}
\newtheorem{lemma}{Lemma}

\title{Notes from Proof Theory by Takeuti}
\author{gajukbhat}
\date{April 2022}

\def\fCenter{\mbox{$\rightarrow$}}

\begin{document}

\maketitle

\section{Chapter 1}
\subsection{Lemma 2.12}
Make the proof regular to get \(P''(a)\). If the proof contains no eigenvariables that are \(a\) or contained in \(t\), then \(P'(a) = P''(a)\).
Let the induction hypothesis be that if the proof contains up to \(n\) eigenvariables that are \(a\) or contained in \(t\), then there's a \(P'(a)\) derived from \(P(a)\) that is a proof of \(\Gamma(a) \rightarrow \Delta(a)\). It's possible to show that the induction hypothesis implies that if the proof contains \(n + 1\) eigenvariables that are \(a\) or contained in \(t\), then there's a \(P'(a)\) derived from \(P(a)\) that is a proof of \(\Gamma(a) \rightarrow \Delta(a)\).
From that the conclusion follows.

\subsection{Definition 2.15}
I want to prove that the definition defines an equivalence relation. Suppose for some \(u_1, \ldots, u_n\) and \(v_1, \ldots, v_n\)
% TODO: Create a macro to make typing this easier.
\[
A\biggl(\frac{u_1, \ldots, u_m}{w_1, \ldots, w_m}\biggr)
\]
and
\[
B\biggl(\frac{v_1, \ldots, v_m}{w_1, \ldots, w_m}\biggr)
\]
are the same. And suppose for some \(x_1, \ldots, x_n\) and \(y_1, \ldots, y_n\)
\[
B\biggl(\frac{x_1, \ldots, x_n}{z_1, \ldots, z_n}\biggr)
\]
and
\[
C\biggl(\frac{y_1, \ldots, y_n}{z_1, \ldots, z_n}\biggr)
\]
are identical.

Later, there's a claim that we can prove by induction on the number of logical symbols in \(A\) that if \(A \sim B\), then \(A \equiv B\) is provable without cut.
I could get it if I took \(A \equiv B\) to mean \(A \rightarrow B\) and \(B \rightarrow A\).
Once I apply any one of the rules of sequent calculus, there's no going back. The formulae only get more complex.
Wait until I learn more about cut elimination and keep going with what I have.

\subsection{Exercise 3.11}
    \begin{description}
        % TODO: See if I really need to add a colon after the item description argument.
        \item[Outermost symbol is \(\supset\):]
        Assume there are \textbf{LJ}-proofs of the sequents \(\neg\neg A \rightarrow A\)
        and \(\neg \neg B \rightarrow B\).
        Let me prove the sequent \(\neg \neg (A \supset B) \rightarrow (A \supset B)\).
        PS-This one's actually easy when I use the result of (1) from Exercise 3.10 and the cut formula.
        I got the rest as well.
%            \begin{prooftree}
%                \Axiom$A, \fCenter\ B$
%                \UnaryInf$A, B \fCenter\ C$
%            \end{prooftree}
    \end{description}
\end{document}
