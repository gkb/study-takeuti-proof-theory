\documentclass{article}

\usepackage[utf8]{inputenc}
\usepackage{amsmath}
\usepackage{mathabx}
\usepackage{mathrsfs}
\usepackage{hyperref}
\usepackage{MnSymbol}
\usepackage[dvipsnames]{xcolor}

\DeclareMathOperator{\Wf}{Wf}
\DeclareMathOperator{\We}{We}
\DeclareMathOperator{\WfWe}{WfWe}
\DeclareMathOperator{\Isom}{Isom}
\DeclareMathOperator{\On}{On}
\DeclareMathOperator{\Le}{Le}
\DeclareMathOperator{\St}{St}
\DeclareMathOperator{\Tr}{Tr}

\newcommand{\RelationLessThan}[2]{(#1^{-1})\text{''}\{#2\}}
\newcommand{\RLessThan}[1]{\RelationLessThan{R}{#1}}

\newtheorem{theorem}{Theorem}

\title{Notes from Introduction to Axiomatic Theory by Takeuti and Zaring}
\author{gajukbhat}
\date{October 2021}

\begin{document}

\maketitle

\section{Chapter 4}
\subsection{Proposition 4.12}
Consider the possibilities for the term \(A\).
\begin{description}
  \item[Case 1] \(A\) is a class symbol.
  \item[Case 2] \(A\) is a free variable.
\end{description}

\section{Chapter 5}
\subsection{Exercises on Page 17}

I should revisit what \(\mathscr{M}(a)\) represents.

PS-That was easy enough.

\section{Chapter 6}
\subsection{Exercises on Page 25}

\subsubsection{Exercise 16}

I'm trying out some \LaTeX{} here. To solve the actual problem itself, I need to
use the definition of equality of classes.

\[
  A = \{z \vert (\exists x) (\exists y) [z = \langle x, y \rangle]\}
\]

\subsubsection{Exercise 20}

I'm experimenting even more with \LaTeX{}.
\begin{align*}
  A \restriction B \\
  A\text{``}B
\end{align*}

\subsection{Exercises on Page 28}

\subsubsection{Exercise 7}

I don't need to think about the existence of \(A_2 \circ A_1\) as a function I
suppose. There's a problem with the matching of domains and ranges of \(A_1\)
and \(A_2\). The inverse of a relation formed from a composition is not the
inverse of the composition. This is very helpful to keep in mind. The usage of
types also makes this clear.

\subsubsection{Exercise 11}
Doesn't \(\mathscr{F}\mathit{n}_2(A) \rightarrow \mathscr{F}\mathit{n}(A)\)?
Indeed, it does.

\subsection{Proposition 6.26}
I know that \(R \WfWe B\). Therefore the following holds.
\[
  (\forall x \in B) \mathscr{M}(B \cap \RLessThan{x}))
\]

\subsection{Proposition 6.31}
I tried to use the definition of \(H \Isom_{R_1, R_2}(A_1, A_2)\).

\subsection{Exercise on Page 34}
We know that \(R_1 \We A_1 \leftrightarrow R_2 \We A_2\). Is it safe to claim
that \((R_2^{-1})\text{''}\{H\text{'}(x)\} =
  H\text{''}(R_1^{-1})\text{''}\{x\}\)?

\begin{align*}
  (R_2^{-1})\text{''}\{H\text{'}(x)\} &= \{y \vert y R_2 H\text{'}x\} \\
  &= \{y \vert \exists! w (y = H\text{'}w \wedge
      H\text{'}w R_2 H\text{'}x) \} \\
  &= \{y \vert \exists! w (y = H\text{'}w \wedge
      w R_1 x) \} \\
  &= \{y \vert \exists! w (y = H\text{'}w \wedge
      w \in R_1^{-1}\text{''}\{x\}) \} \\
  &= H\text{''}(R_1^{-1})\text{''}\{x\}
\end{align*}

Indeed! The problem is a bit unusual in that it doesn't propose this result
directly.

\subsection{Exercises on Page 42}
\subsubsection{Exercise 1}
The following result is useful.
\[
  \alpha = \cup (\alpha) \vee \alpha = \cup (\alpha) + 1
\]

\section{Chapter 7}

\subsection{Proposition 7.33}
If \(\omega \in K_1\), then \(\omega \in \omega\).

\subsection{Exercises on Page 45}
\subsubsection{Exercise 5}
\(A - \alpha \neq 0\) implies \(\alpha \in A\) and for all \(\beta \in (A -
\alpha)\), \(\alpha \subseteq \beta\).

\subsection{Exercises on Page 50}
\subsubsection{Exercise 4}
\begin{theorem}
Assume \(R \We A\).
\[
  K = \{f \vert (\exists a \in A) (f \mathscr{F} a \wedge
    (\forall x \in \RLessThan{a}))(f\text{'}x = G\text{'}(f \restriction
        \RLessThan{x})))) \}
\]

  Let \(F = \bigcup (K)\). Then
  \begin{itemize}
    \item \(F \mathscr{F} A\)
    \item \((\forall a \in A) (F \text{'}a =
      G\text{'}(F\text{''}(\RLessThan{a}))))\)
    \item \((F_1 \mathscr{F} A) \wedge
     ((\forall a \in A) (F_1 \text{'}a =
      G\text{'}(F_1 \text{''}(\RLessThan{a})))))
      \rightarrow F_1 = F\)
  \end{itemize}
\end{theorem}

\subsection{Proposition 7.47}
Choose \(x \in (A - B)\) such that \((A - B) \cap \RLessThan{x}) = 0\). In other
words, \(x\) is the \(R\)-minimal element in \((A - B)\).

\subsection{Proposition 7.51}
If I show that \(\mathscr{W}(F)\) is R-transitive, then I can prove that it is a
bijection from \(\On\) to \(A\).

That follows from the definition of \(G\).

Can I show that \(F\text{''}(\alpha) = \RLessThan{F\text{'}(\alpha)} \cap A\)? I
already have \(\RLessThan{F\text{'}(\alpha)} \cap A \subseteq
F\text{''}(\alpha)\). Indeed, what Takeuti and Zaring show in
their proof amounts to proving that \(F\text{''}(\alpha) \subseteq
\RLessThan{F\text{'}(\alpha)} \cap A\).

\subsection{Proposition 7.56}
\begin{itemize}
  \item The well-ordering is easy to prove once I get the foundedness.
    \[
      (\forall x) ((x \neq 0 \wedge x \subseteq \On^2) \rightarrow
          (\exists y)(x \cap \RelationLessThan{\Le}{y} = 0))
    \]

  And I got the foundedness.
\end{itemize}

\subsection{Exercises on Page 62}
The definition of \(\minusdot\) was interesting. I made a note on the book.

\subsubsection{Exercise 4}
I tried to do this by induction on \(n\). That leads me to want the proof of
this statement.

\[
  (m + 1) = (1 + m)
\]

Is there a simple way to get to this? I have to use induction again. It's true
for \(m = 0\) by definition. Assume it's true for \(m \le n\). Then,
\begin{align*}
  ((n + 1) + 1) &= ((1 + n) + 1) \\
                &= (1 + (n + 1) &\text{by associativity.}
\end{align*}
Now the rest of the proof is straightforward.

\subsubsection{Exercise 10}
The base case and non-limit case are straightforward. Assume \(\beta \in
K_{II}\) and for all \(\alpha \le \zeta < \beta\), there exists \(\delta\) such
that \(\alpha + \delta = \zeta\).

\[
  \delta_E = \bigcup \{\delta \vert \exists \zeta ((\alpha \le \zeta < \beta)
      \wedge (\alpha + \delta = \zeta))\}
\]

The claim is that \(\alpha + \delta_E = \beta\). Clearly for all \(\gamma \in
\delta_E\), there exists \(\delta\) such that \(\gamma \subseteq \delta\).
Therefore, \(\alpha + \delta_E \subseteq \beta\).

If I just prove that \(\delta_E \in K_{II}\), I think I'll be done. That's
probably easy. Consult the right steps though.

PS-From the definition, we have the following.
\[
  \forall \zeta ((\alpha \le \zeta < \beta) \rightarrow \\
      (\alpha + \zeta) \subseteq (\alpha + \delta_E))
\]

The result is immediate.

\subsection{Exercises on Page 66}
\subsubsection{Exercise 3}

Start with the proof of \(mn = nm\). The result is clear for \(n=0,1\).

Assume it's true for \(n\). Then the following holds.
\begin{align*}
  m(n + 1) &= mn + m \\
    &= nm + m
\end{align*}

I suppose I can prove my induction again that \((n+1)m = nm + m\). It's clear for
\(m = 0\). Assuming it's true for \(m\), the following holds.
\begin{align*}
  (n + 1)(m + 1) &= (n + 1)m + (n + 1) \\
    &= nm + m + n + 1 \\
    &= (nm + n) + (m + 1) \\
    &= n(m + 1) + (m + 1)
\end{align*}

The result is immediate.

Now consider the right distributive law. Again, we use induction on \(k\). For
\(k = 0\), the result is clear. Assuming the result holds up to \(k\), the
following is true.
\begin{align*}
  (m + n)(k + 1) &= (m + n)k + (m + n) \\
    &=(mk + nk) + (m + n) \\
    &=m(k + 1) + n(k + 1)
\end{align*}

\subsubsection{Exercise 4}
The counterexample would be \(\alpha = \omega\) and \(\gamma = \omega 2\).
\(\alpha\gamma = \omega\omega 2 \neq \omega 2\).

\subsection{Proposition 8.44}
Suppose there are two different expansions.
\begin{align*}
  \beta &= \alpha^{\beta_{n_1}} \gamma_{n_1} + \cdots
      \alpha^{\beta_{0}} \gamma_{0} \\
        &= \alpha^{\delta_{n_2}} \eta_{n_2} + \cdots
      \alpha^{\delta_{0}} \eta_{0}
\end{align*}

If you assume \(\beta_{n_1} > \delta_{n_2}\), we end up with a contradiction
using Proposition 8.43. Therefore \(\beta_{n_1} = \delta_{n_2}\). Using the same
proposition, I can prove that \(\eta_{n_2} = \gamma_{n_1}\). How do I deal with
the difference between \(n_1\) and \(n_2\)?

PS-It's safe to use induction.

\section{Chapter 9}
\subsection{Proposition 9.10}
(a) Let me prove the inductive step for non-limit ordinals. Clearly, it's not true
that \(R_1 \alpha \nsubseteq R_1 (\alpha + 1)\).

Suppose that \(R_\alpha\) satisfies the conditions \(\Tr(R_\alpha)\) and
\(\St(R_\alpha)\). What can we say about \(R_{\alpha + 1}\)?

If \(y \in R_{\alpha + 1}\), then \(y \subseteq R_\alpha\). Can we claim that
\(y \subset R_{\alpha + 1}\) and \(\mathcal{P}(y) \subseteq R_{\alpha + 1}\) or
equivalently \(w \subseteq y \rightarrow w \in R_{\alpha + 1}\)?
This is true for \(\alpha = 0\).

See if it's a good idea to look at the powerset relationship.

Look at \(x \in y\). We have \(y \subseteq R_\alpha \rightarrow x \in
R_\alpha\). Because of transitivity \(x \subseteq R_\alpha\) and therefore \(x
\in R_{\alpha + 1}\), Thus \(y \in R_{\alpha + 1} \rightarrow x \in R_{\alpha +
1}\).

Likewise \(w \subseteq y\) implies \(w \subseteq R_\alpha\) and therefore
\(w \in R_{\alpha + 1}\). I didn't need to use the second part of the
super-transitivity of \(R_\alpha\) at all. Now that's curious.

Take the limit ordinals now.
\end{document}
